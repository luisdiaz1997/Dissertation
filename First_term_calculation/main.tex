\documentclass{article}
\usepackage{xcolor}
\usepackage{amsmath}
\usepackage{amssymb}

\title{Expectation Term Calculation}
\date{September 2025}

\begin{document}

\maketitle

\section{Expansion of Expected Log Likelihood}


We begin with the expected log-likelihood term
\[
\mathcal{L}_1
= \sum_{j=1}^D \sum_{i=1}^N
\mathbb{E}_{q(F_{i\cdot})} \!\left[ \log p(Y_{ij} \mid F_{i\cdot}) \right],
\]

For a Poisson observation model:

\[
p(Y \mid \nu \lambda) = \frac{\nu\lambda^Y e^{-\nu\lambda}}{Y!}
\]

The log-likelihood is:

\begin{align*}
\log p(Y_{ij} \mid \nu_i\lambda_{ij}) &= Y_{ij} \log \nu_{i} \lambda_{ij} - \nu_{i}\lambda_{ij} - \log (Y_{ij}!)\\
&= Y_{ij} \log  \lambda_{ij} - \nu_{i}\lambda_{ij} + Y_{ij} \log \nu_{i} - \log (Y_{ij}!)
\end{align*}


In the NSF parameterization we set
\[
\lambda_{ij} \;=\; \sum_{\ell} W_{j\ell} e^{F_{i\ell}},\qquad W_{j\ell}\ge 0.
\]

Then the expected log-likelihood becomes:

\begin{align*}
\mathbb{E}_{q(F_{i\cdot})}\!\left[\log p(Y_{ij}\mid F_{i\cdot})\right]
&= Y_{ij}\, \mathbb{E}_{q(F_{i\cdot})}\!\left[ \log \sum_{\ell} W_{j\ell} e^{F_{i\ell}} \right]
\;-\; \nu_{i}\mathbb{E}_{q(F_{i\cdot})}\!\left[\sum_{\ell} W_{j\ell} e^{F_{i\ell}} \right]
\;-\; \log(Y_{ij}!) + Y_{ij} \log \nu_{i}\\
&= Y_{ij}\, \mathbb{E}_{q(F_{i\cdot})}\!\left[ \log \sum_{\ell} W_{j\ell} e^{F_{i\ell}} \right]
\;-\; \nu_{i}\sum_{\ell} W_{j\ell}\, \mathbb{E}_{q(F_{i\ell})}\!\left[ e^{F_{i\ell}} \right]
\;-\; \log(Y_{ij}!)+ Y_{ij} \log \nu_{i} .
\end{align*}


Since \( F_{i\ell} \sim \mathcal{N}(\mu_{i\ell}, \sigma_{i\ell}^2) \), we use the moment-generating function of the Gaussian to evaluate:

\[
\mathbb{E}_{q(F_{i\ell})} \left[ e^{F_{i\ell}} \right]
= \exp\left( \mu_{i\ell} + \frac{1}{2} \sigma_{i\ell}^2 \right)
\]

This becomes:

\[
\mathbb{E}_{q(F_{i \cdot})} \left[ \log p(Y_{ij} \mid F_{i \cdot}) \right]
= Y_{ij} \, \mathbb{E}_{q(F_{i \cdot})} \left[ \log \sum_\ell W_{j\ell} e^{F_{i\ell}} \right]
- \nu_{i} \sum_\ell W_{j\ell} \, \exp\left( \mu_{i\ell} + \frac{1}{2} \sigma_{i\ell}^2 \right)
- \log (Y_{ij}!)+ Y_{ij} \log \nu_{i}
\]


\section{Bounds for the expectation of the log-sum-exp term}

Assume a mean-field variational posterior across factors,
\(
q(F_{i\cdot})=\prod_{\ell} q(F_{i\ell}).
\)

% \paragraph{Iterated expectation (nesting).}
% Fix an index \(\ell\). Conditioning on \(F_{i,\setminus \ell}\) and integrating over \(F_{i\ell}\) gives
% \begin{align*}
% \mathbb{E}_{q(F_{i\cdot})}\!\left[ \log \sum_{m} W_{jm} e^{F_{im}} \right]
% &= \mathbb{E}_{q(F_{i,\setminus \ell})}\!\left[
%   \mathbb{E}_{q(F_{i\ell})}\!\left[
%     \log\!\Big(W_{j\ell}e^{F_{i\ell}} + \!\!\sum_{m\neq \ell} W_{jm} e^{F_{im}}\Big)
%     \,\Big|\, F_{i,\setminus \ell}
%   \right]
% \right].
% \end{align*}
% Define the (random, but \emph{fixed} inside the inner expectation) offset
% \[
% \eta_{ij}^{(\ell)}(F_{i,\setminus \ell})
% \;:=\;
% \sum_{m\neq \ell} W_{jm} e^{F_{im}} \;>\; 0.
% \]
% Then
% \begin{align*}
% \mathbb{E}_{q(F_{i\ell})}\!\left[
%     \log \big( W_{j\ell}e^{F_{i\ell}} + \eta_{ij}^{(\ell)} \big)
%   \,\Big|\, F_{i,\setminus \ell}
% \right]
% &= \log \eta_{ij}^{(\ell)} \;+\;
% \mathbb{E}_{q(F_{i\ell})}\!\left[
%     \log \Big( 1 + e^{\,F_{i\ell} + \log W_{j\ell} - \log \eta_{ij}^{(\ell)}} \Big)
%   \right] \\
% &= \log \eta_{ij}^{(\ell)} \;+\;
% \mathbb{E}_{q(F_{i\ell})}\!\left[
%     \mathrm{softplus}\!\Big( F_{i\ell} + \log W_{j\ell} - \log \eta_{ij}^{(\ell)} \Big)
%   \right].
% \end{align*}

\paragraph{Jensen sandwich bounds.}
Let \(g(z)=\log\!\sum_{m} W_{jm} e^{z_m}\) (convex) and \(S=\sum_m W_{jm} e^{F_{im}}>0\).
Then
\[
\boxed{~
\log \sum_{m} W_{jm}\, e^{\,\mathbb{E}[F_{im}]}
\;\le\;
\mathbb{E}\!\left[\log \sum_{m} W_{jm} e^{F_{im}}\right]
\;\le\;
\log \sum_{m} W_{jm}\, \mathbb{E}\!\left[e^{F_{im}}\right]
~}
\]
(the left inequality uses convexity of \(g\), the right uses concavity of \(\log\): \(\mathbb{E}[\log S]\le \log\mathbb{E}[S]\)).
For Gaussian marginals this becomes
\[
\log \sum_{m} W_{jm}\, e^{\mu_{im}}
\;\le\;
\mathbb{E}\!\left[\log \sum_{m} W_{jm} e^{F_{im}}\right]
\;\le\;
\log \sum_{m} W_{jm}\, \exp\!\big(\mu_{im}+\tfrac12\sigma_{im}^2\big).
\]

\section{Approximating this term}





\end{document}


